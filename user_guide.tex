\documentclass[11pt]{article}
\usepackage[english]{babel}
\usepackage[colorlinks=true, allcolors=blue]{hyperref}

    \title{\textbf{User Guide}}
    \author{Andrei Suba\\
    Computer Science Student,\\
	West University of Timișoara,\\
	Timișoara, Romania\\
	\textbf{Email:} \texttt{andrei.suba00@e-uvt.ro}
    }
    \date{}
    
    \addtolength{\topmargin}{-3cm}
    \addtolength{\textheight}{3cm}

\begin{document}

\maketitle
\thispagestyle{empty}

\clearpage

\tableofcontents

\clearpage

\section{Welcome!}
This user guide was created for the Individual Project subject at my university and it's meant
to describe my work for the past few months.\\
The project it's about a new linux distribution based on \href{https://archlinux.org/}{Arch Linux} which
follows the same design of minimalism with regards to its lightweight functioning and flexibility.\\
This linux distribution comes with a lightweight tiling window manager created by the \href{https://suckless.org/}{Suckless Team} named \texttt{dwm} and other tools like \texttt{dmenu}, \texttt{st} and \texttt{slock}.\\
The implementation of their software and others will be explained thoroughly in the next sections.\\
\newline
Disclaimer: this work began a long time ago just as a personal project out of curiosity after going down the 
rabbit hole of minimalism setups. Shoutouts to \href{https://lukesmith.xyz/}{Luke Smith}!

\section{Basic goals and principles of this system}

\noindent\textbf{Economy}
\newline
Programs should be simple and light on system resources and highly e xtensible.
Because of this, many are terminal or small ncurses programs that have all the magic inside
of them.
\newline
\newline
\noindent\textbf{Keyboard}
\newline
All terminal programs (and other programs) use vim keys when possible. Your hands could never need to leave the keyboard, increasing the proficiency once the user gets used to it.
\newline
\newline
\noindent\textbf{Decentralization}
\newline
This system is a web of small, modifiable and replaceable programs that users can easily customize.

\section{Status bar}

The status bar is created through scripts which are to be added in the configuration file of \texttt{Dwmblocks}.\\
The output of the scripts is the one displayed. The status bar is composed as
follows (from left to right):
\begin{itemize}
	\item \textbf{RAM}: it displays the currently utilized RAM. 
	\item \textbf{Battery}: displays the battery level 
	\item \textbf{Volume}: it displays the volume as a percentage 
	\item \textbf{Internet}: if there is no internet connection,
	it displays just the wifi icon. If there is a connection it displays based on the connection type: 
	\begin{itemize}
		\item \textbf{Ethernet}: an ethernet icon followed by "Eth"
		\item \textbf{Wifi}: an wifi icon followed by the SSID (wifi name)
	\end{itemize}
	\item \textbf{Date}: date and time
	\item \textbf{Help}: access to this user guide
\end{itemize}

\subsection{Interactions}

The status bar has a patch that allows it's modules to be clickable, that
meaning that if you hover your mouse over any of the modules and press either
\texttt{Left Click}, \texttt{Right Click} or \texttt{Shift + Left Click}, these
actions will happen:
\begin{itemize}
	\item \texttt{Left Click}: sends a desktop notification with additional
	information 
	\item \texttt{Right Click}: depending on the script, the following will 
	happen:
	\begin{itemize}
		\item \textbf{RAM}: program \texttt{htop} will open for a more
		detailed view of RAM consumption, CPU load and processes
		\item \textbf{Volume}: program \texttt{pavucontrol} will open to
		control more aspects regarding audio devices
		\item \textbf{Internet}: program \texttt{nmtui} will open to allow
		the user to connect to a wifi network
		\item \textbf{Help}: will open this user guide
	\end{itemize}
	\item \texttt{Shift + Left Click}: it will open up in vim the script that
	is producing the output
\end{itemize}

\section{Key bindings}

In the examples below, \texttt{Mod} refers to the "Windows key" or Super Key.
\newline
\begin{itemize}
	\item \texttt{Mod} + \texttt{Enter} : open st teminal
	\item \texttt{Mod} + \texttt{Q} : exit application
	\item \texttt{Mod} + \texttt{Shift} + \texttt{Q} : restart dwm
	\item \texttt{Mod} + \texttt{Shift} + \texttt{L} : lock screen
	\newline
	\newline
	\underline{NOTE}: in order to unlock the screen, you will need
	to just type your user's password and press \texttt{Enter}. If you
	mess up the password just press \texttt{Enter} and type it again
	until you succeed.
	\item \texttt{Mod} + \texttt{W} : open firefox (web browser)
	\item \texttt{Mod} + \texttt{Shift} + \texttt{w} : open lynx (terminal web browser)
	\item \texttt{Mod} + \texttt{D} : launch dmenu (to run programs)
	\item \texttt{Mod} + \texttt{O} : launch lf (terminal based file manager)
	\item \texttt{Mod} + \texttt{Shift} + \texttt{O} : launch thunar (graphical file manager)
	\item \texttt{Mod} + \texttt{-} : decrease volume by 5\%
	\item \texttt{Mod} + \texttt{+/=} : increase volume by 5\%
	\item \texttt{Mod} + \texttt{Shift} + \texttt{-} : decrease brightness by 5\%
	\item \texttt{Mod} + \texttt{Shift} + \texttt{+/=} : increase brightness by 5\%
	\item \texttt{Mod} + \texttt{J/K} : browse through applications in the same window
	\item \texttt{Mod} + \texttt{H/L} : increase/decrease window size
	\item \texttt{Mod} + \texttt{N/B} : go next/back through tags
	\item \texttt{Mod} + \texttt{Tab} : switch between current and previous tag
	\item \texttt{Mod} + \texttt{Space} : switch focus of window
	\item \texttt{Mod} + \texttt{1/2/3/4/5} : switch to tag 1/2/3/4/5
	\item \texttt{Mod} + \texttt{Shift} + \texttt{1/2/3/4/5} : move application to tag 1/2/3/4/5	
\end{itemize}

\section{Tips}

In order to select the way the screens are arranged and display,
the script \texttt{displayselect} may be useful.\\
Run it from terminal or with \texttt{dmenu} (the application runner).

\section{Conclusion}
The project successfully combined the minimalist philosophy of the suckless community with the flexible and powerful Arch
Linux operating system. By carefully selecting and installing a range of software from the suckless community and other software that shares the same philosophy, a custom Linux distribution emerged that prioritized simplicity, efficiency, and customization. This project serves as a confirmation to the value of minimalist and efficient design.\\
\newline
This project was a rewarding and satisfying experience for me because it\\
integrates different values that I care for like minimalism and open-source\\
software. It challenged me to think creatively and critically about different approaches to accomplish this result.


\end{document}

